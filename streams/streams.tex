\documentclass[pdftex,11pt]{article}
\usepackage{fancyvrb}
\usepackage{color}
\usepackage[T1]{fontenc}
\usepackage[light,math]{iwona}
\usepackage{multicol}
\usepackage{hyperref}

\addtolength{\oddsidemargin}{-.875in}
\addtolength{\evensidemargin}{-.875in}
\addtolength{\textwidth}{1.75in}
\addtolength{\topmargin}{-.75in}
\addtolength{\textheight}{1.0in}

%\setlength{\parindent}{0pt}

\author{Vishal Talwar}
\title{Streams in Scheme\footnote{Material based on \S3.5 of SICP}}

\begin{document}

\maketitle

\DefineVerbatimEnvironment%
  {scheme}{Verbatim}
  {numbers=left,numbersep=5pt,
   frame=lines,framerule=.3mm,framesep=8pt,commandchars=\\\{\}}

\DefineVerbatimEnvironment%
  {interaction}{Verbatim}
  {framesep=8pt,framerule=.1mm,,commandchars=\\\{\}}

\newcommand{\fiblist}{{\tt fibs}\ }
\newcommand{\fiblistf}[1]{{\tt (fibs-from #1)}}
\newcommand{\scons}{{\tt s-cons}\ }
\newcommand{\scar}{{\tt s-car}\ }
\newcommand{\scdr}{{\tt s-cdr}\ }
\newcommand{\sref}{{\tt s-ref}\ }
\newcommand{\sgen}{{\tt s-generate}\ }
\newcommand{\smap}{{\tt s-map}\ }
\newcommand{\smerge}{{\tt s-merge}\ }
\newcommand{\sfilt}{{\tt s-filter}\ }
\newcommand{\sinter}{{\tt s-interleave}\ }
\newcommand{\cons}{{\tt cons}\ }
\newcommand{\car}{{\tt car}\ }
\newcommand{\cdr}{{\tt cdr}\ }
\newcommand{\lref}{{\tt list-ref}\ }
\section{Prerequisites}
Before reading this, you should be familiar with:
\begin{itemize}
\item list operations: \car, \cdr, \cons, \lref, etc.
\item higher-order functions: map, filter, 
\item recursion
\end{itemize}
Thats it! Now, we've got some work to do, so put on your sweatbands
and read on.
\section{Let's define ``stream''}
A stream is a potentially infinite list. This concept may sound a
little out there at first, but we reason about the infinite all the
time. For example, say you ask somebody a question. You can consider
their response a stream. You don't know how long they might talk for
(and let's face it, some people do seem to go on forever). More
realistically, consider the positive numbers, a set very convenient
for us to represent as an infinite list:
\begin{interaction}
1 2 3 4 5 6 7 8 ...
\end{interaction}
That's nice, but this isn't Scheme yet. How about this:
\begin{interaction}
'(1 2 3 4 5 6 7 8 ...)
\end{interaction}
Is our job done? No, because '\ldots' has no meaning in Scheme,
although its presence makes a world of difference to us. It implies
that you should continue the pattern to generate more elements of the
stream. What's the pattern? Well, we know the stream begins with a
1. Given an element in the stream, how do we get the next one?
\begin{scheme}
(define (next-integer n)
  \underline{                })
\end{scheme}
That's simple. We just need to add one to the number to get the next
number. This combination of a starting element and a function to get
from one element to the next is just one way of thinking of a
stream. You'll soon see that this is not the most general way of
thinking of them, but it's a very useful one nonetheless. You'll also
see that in order to work with streams in a finite amount of computer
memory, we'll have to employ a tactic similar to using the '\ldots' to
represent ``the rest of the stuff''.

Wait. Before we go on, there's one stone that has been left unturned
in all of this. We said at the very beginning that a stream is {\it
  possibly} infinite, so when is a stream ever not, and why would we
use a stream instead of a list for that? This question will be brushed
aside for the rest of this document, because it actually is a lot more
complex than it may seem. In fact, the functions we'll define will
blatantly ignore the fact that streams can be finite. Not to worry, a
better distinction between streams and lists will be made later on,
and we'll see just how cool and useful infinite streams can be.

\section{Let's (try to) make a stream}

How do you like the Fibonacci numbers? Do they form a stream? Well, we
can start by writing them out:
\begin{interaction}
'(0 1 1 2 3 5 8 ...)
\end{interaction}
That's nice, but this isn't legal Scheme code either, although it
certainly {\it looks} like it could be a stream. Can we figure out the
pattern behind the '\ldots' in this case? You've most likely written a
function that generates the $n$th Fibonacci number at some point, so
let's do it just once more:
\begin{scheme}
(define (fib n)
  (cond ((= n 1) \underline{  })
        ((= n 2) \underline{  })
        (else (+ (\underline{          })
                 (\underline{          })))))
\end{scheme}
That gives us a way to create more and more Fibonacci numbers, so we
must be closing in on a stream now! Let's try to create a stream of
Fibonacci numbers, {\tt fibs}, using the function we just defined:
\begin{scheme}
(define fibs
  (cons (fib 0)
        (cons (fib 1)
              (cons (fib 2)
                     ...?
\end{scheme}
Hmm, that didn't work. When in the world can we stop writing this
function? We ran into this very problem trying to write out the stream
of positive integers as a list using {\tt '()}. The issue is that we
want \fiblist to be an infinite list of the Fibonacci numbers without
having to explicity write out every member like we had started to. No
problem. We have a great way of dealing with this all-too-common task,
namely, recursion! Are you in the mood for some recursion?  If so,
read on:
\begin{scheme}
(define fibs
  (cons (fib 0) fibs))...?
\end{scheme}
That doesn't seem to be going anywhere either. Even though we got the
first element, {\tt (fib 0)}, if we \cons that back onto \fiblist
(line 2), what happens?  Let's take a look:
\begin{interaction}
   fibs
=> (cons (fib 0) fibs)
=> (cons (fib 0) (cons (fib 0) fibs))
=> (cons (fib 0) (cons (fib 0) (cons (fib 0) fibs)))
   ...
\end{interaction}
We just get a bunch of {\tt (fib 0)}'s! The problem might seem
obvious: \fiblist does not take any arguments, so why should we expect
it to do anything different each time we expand it as we did above. In
general, no matter how many {\tt fib}s we \cons on before the
recursive call, we won't get what we want. Let's redefine the function
to accept an argument $n$ and generate all Fibonacci numbers beginning
from the $n$th onwards. This sounds like it could be helpful, so let's
try it:
\begin{scheme}
(define (fibs-from n)
  (cons (fib n)
        (fibs-from (+ n 1))))
\end{scheme}
Don't pat yourself on the back just yet. Does this do what we wanted?
Expand \fiblistf{0}:
\begin{interaction}
(fib-from 0)
(cons (fib 0) (fib-from 1))
(cons (fib 0) (cons (fib 1) (fib-from 2))
(cons (fib 0) (cons (fib 1) (cons (fib 2) (fib-from 3))))
   ...
\end{interaction}
Ok, pat yourself on the back now. This does indeed look like an
infinite list of Fibonacci numbers. Have you tried this in a Scheme
interpreter? If not, try it now:
\begin{interaction}
(fib-list 0)
=> ...
\end{interaction}
In short, you're not likely to see anything except an error message
(if you wait long enough). What our recursive definition missing that
recursive definitions usually have? A {\bf base case}. \fiblistf{0}
never stops expanding because there's no reason it should! We didn't
give it cause to stop recursing, because we absolutely {\it want} it
to be an infinite list.

This should worry you. We were so close to calling the output of
\fiblistf{n} a stream, but the barrier we just hit seems to be
insurmountable. The good news is that it isn't. We will get to the
magic shortly, but first, be sure to take a break or go over the
examples again if it is unclear what a {\bf stream} is or how
{\tt fibs-from} is like a stream, but not {\it quite}.

\section{Let's pretend we have streams}

\subsection{Infinite recursion}

Let's recap what went wrong when we tried to define a stream for the
Fibonacci sequence. Try to say this in your own words before reading
on.

We'd really like to say that \fiblistf{n} is {\tt (fib n)} consed onto
\fiblistf{(+ n 1)}. This way, we can think of streams almost the same
way we think of lists. However, this definition is {\it infinitely}
recursive, because \fiblistf{(+ n 1)} will call \fiblistf{(+ n 2)} and
that will, in turn, call \fiblistf{(+ n 3)}, and so on ad
infinitum. For now, let's ignore this problem and pretend as though
streams {\bf just work}.

\subsection{Diving into streams}

So, now what? What can we do with these imaginary streams? Well, as we
just said, it would be nice if we could treat them like lists. We'd
like to \car and \cdr them in order to pull elements out of them, for
starters. Let's imagine \scar and \scdr functions that operate on
streams the same way \car and \cdr do on lists, and test them on a
couple of predefined streams. We'll call the stream of positive
numbers {\tt ints} and the stream of Fibonacci numbers {\tt fibs}.

\begin{interaction}
\textit{;; ints first}
(s-car ints)
=> 1
(s-car (s-cdr ints))
=> 2
(s-car (s-cdr (s-cdr ints)))
=> 3
\end{interaction}
\begin{interaction}
\textit{;; on to fibs}
(s-car fibs)
=> 0
(s-car (s-cdr fibs))
=> 0
(s-car (s-cdr (s-cdr (s-cdr fibs))))
=> 2
(s-car (s-cdr (s-cdr (s-cdr (s-cdr (s-cdr fibs))))))
=> 5
\end{interaction}

Notice that we refrained from looking at the output of {\tt (s-cdr
  {\it stream})} directly. Don't worry about it for now. What we see
is that we can \scdr down a stream and pull elements out with
\scar. This is a breakthrough! Convince yourself why before
continuing. Hint: can we now work with infinite lists? It's (finally)
time for some real exercises. We can build up some more stream
operators out of the two we already have. Let's start simple.

\begin{enumerate}
\item Define a function {\tt s-cadr}, which does to streams what {\tt
    cadr} does to lists:
\begin{scheme}
(define (s-cadr s)
  (\underline{      } (\underline{      } s)))
\end{scheme}
That was easy, right? Don't worry, there's more blanks in the next
one.

\item Define a function {\tt s-generate} that takes a stream and a
  number $n$ and returns a {\bf list} of $n$ elements from that
  stream. This is an extremely useful function, so let's see it in
  action:
\begin{interaction}
(s-generate ints 0)
=> ()
(s-generate ints 10)
=> (1 2 3 4 5 6 7 8 9 10)
(s-generate fibs 10)
=> (0 1 1 2 3 5 8 13 21 34)
\end{interaction}
\begin{scheme}
(define (s-generate s n)
   (if (\underline{      }) \underline{  }
       (cons (\underline{     } s)
             (s-generate \underline{      } \underline{      }))))
\end{scheme}

\end{enumerate}

We're all set now to turn the imagined into the real.

\section{Let's touch base with reality}

Recall the infinite recursion that was the root of all our
problems. What if there was a way to tell Scheme:
\begin{quote}
  Hold on! Don't be so eager to compute the next element in the stream
  (i.e. the next Fibonacci number)! When I need it, I'll ask you for
  it, but you must promise me you'll compute it when I {\it do} ask.
\end{quote}
If Scheme agreed to this, would we be able to define streams? Yes! To
make this happen, let's define a new type of \cons called \scons
\footnote{\scons is defined in {\tt streams.scm}, so be sure to load
  the file before attempting to use it.} . This is a very special
\cons because when we say
\begin{interaction}
(s-cons thing1 thing2)
\end{interaction}
Scheme promises that it will ignore {\tt thing2} until we ask for it,
and no sooner. What we are essentially saying is that
\begin{quotation}
  Infinite recursion allows us to model an infinite stream, {\it
    provided that} we have some way of delaying the recursion until we
  ask for it to happen.
\end{quotation}
Another way to look at it is that
\begin{quotation}
  A list exists all at once. A stream is uncovered element by element
  as we ask for them.
\end{quotation}
This might be clear as daylight or it might be very confusing. Let's
assume it's the latter.  What does it even mean to ``ask'' Scheme for
an element?  Let's take a shot in the dark and try to create an
extramely simple stream, the stream of ones, and ``ask'' for its
elements:
\begin{scheme}
(define ones
  (s-cons \underline{  } \underline{     }))
\end{scheme}
Wow. Is that all? Does this even work? Let's try some of the
operations we're familiar with.
\begin{interaction}
(s-car ones)
=> 1
(s-car (s-cdr ones))
=> 1
(s-generate ones 10)
=> (1 1 1 1 1 1 1 1 1 1)
\end{interaction}
Ah, so that's what we mean by ``asking''! We ask for elements when we
manipulate the stream to pull them out like we just did. But where did
the infinite recursion disappear to? Did Scheme suddenly get lazy and
take a break from our infinite antics?  Not likely. It's merely
upholding the request we made just a minute ago and waiting until we
ask for the rest of the stream with an \scdr before attempting to
compute any more of it. Now, with the help of \scons, we can make our
own streams. Our example with {\tt ones} was so simple we only used a
single \scons, but are we restricted to using a single \scons to make
streams? Certainly not. Let's try to come up with a useful stream that
uses more than one of them, like the stream of alternating 0's and
1's:
\begin{scheme}
(define zeroes-ones
  (s-cons \underline{  }
          (s-cons \underline{  }
                  zeroes-ones)))
\end{scheme}
Nice, that one proves we can use \scons in any situation we use
\cons. The same way a \cons creates a new list, \scons creates a new
{\bf stream}.

Remember our troubles with the Fibonacci sequence above? The
last thing we tried was something like this:
\begin{scheme}
(define (fibs-from n)
  (cons (fib n)
        (fibs-from (+ n 1))))
\end{scheme}
That's right, we were using \cons because we didn't know any
better. We have a better tool now: \scons. This should be easy to
fix. How about the following:
\begin{scheme}
(define (fibs-from n)
  (\underline{       } (fib n)
        (fibs-from (+ n 1))))
\end{scheme}
That most definitely looks right, but how can we get the entire
Fibonacci sequence ({\tt fibs} from before) without that pesky
argument $n$ getting in the way? Give it a go:
\begin{scheme}
(define fibs \underline{                       })
\end{scheme}
Treat yourself to an expensive meal at your favourite restaurant. You
just created your first {\it real} stream, so swap out your sweatbands
for some fresh ones and read on.

\subsection{Round 1: Simple Streams}

Let's define a few streams from scratch, just like we did with {\tt
  fibs} and {\tt ones}.
\begin{enumerate}
\item Define the stream {\tt (ints-from n)} which is the stream of all
  integers starting from $n$ on. Then, using this stream, define a new
  stream {\tt ints} of {\it all} positive integers (1,2,3,\ldots).
\begin{interaction}
(s-car (ints-from 5))
=> 5
(s-car (s-cdr (ints-from 5)))
=> 6
(s-generate (ints-from 5) 10)
=> (5 6 7 8 9 10 11 12 13 14)
\end{interaction}
\begin{scheme}
(define (ints-from n)
  (s-cons \underline{  }
          \underline{                  }))

(define ints \underline{               })
\end{scheme}
\item Good so far? That couldn't have been any harder than {\tt fibs},
  because they're nearly the same! So is this one. Define a pair of
  streams, {\tt evens-from} and {\tt evens}, that define a stream of
  even numbers in a similar fashion to the last problem.
\begin{scheme}
(define (evens-from n)
  (s-cons \underline{  }
          \underline{                   }))

(define evens \underline{               })
\end{scheme}
\end{enumerate}
Excellent. If you haven't yet worked up a sweat, move on to the next
round. Otherwise, take a shower and call it a day.

\subsection{Round 2: Simple Stream Operations}

\begin{enumerate}
\item This is a warmup that will come in handy later. Define the
  following functions:
  \begin{itemize}
  \item {\tt (id n)}: returns $n$ (does nothing to input)
  \item {\tt (double n)}: returns $2n$
  \item {\tt (square n)}: returns $n^2$
  \item {\tt (pow2 n)}: returns $2^n$
  \item {\tt (multiple? m n)}: returns {\tt \#t} if $m$ is a multiple
    of $n$, {\tt \#f} otherwise.
  \item {\tt (even? n)}: returns {\tt \#t} if $n$ is even, {\tt \#f}.
  \end{itemize}
\begin{scheme}
(define (double n)
                                      )
(define (square n)
                                      )
(define (pow2 n)
                                      )
(define (multiple? m n)

                                      )
\textit{;; Version 1: doesn't use multiple}
(define (even? n)

                                      )
\textit{;; Version 2: uses} multiple
(define (even? n)
                                      )
\end{scheme}
\item We've already seen that \sgen is a rather handy function. In
  fact, we can think of it as converting part of a stream into a
  list. Can we think of other think of other things to do with
  streams? How about the stream equivalent of \lref? Define away!
\begin{interaction}
(s-ref ints 0)     ;; \underline{\textit{0}} 1 2 3 4 5 6 7 ...
=> 1
(s-ref ints 4)     ;; 1 2 3 4 \underline{\textit{5}} 6 7 ...
=> 5
(s-ref fibs 0)     ;; \underline{\textit{0}} 1 1 2 3 5 8 ...
=> 0
(s-ref fibs 4)     ;; 0 1 1 2 \underline{\textit{3}} 5 8 ...
=> 3
\end{interaction}
\begin{scheme}
\textit{;; Version 1: Don't use s-generate}
(define (s-ref s n)



                                       )

\textit{;; Version 2: Use {\tt s-generate} and {\tt list-ref} (mind your indices!)}
(define (s-ref s n)


                                       )
\end{scheme}
That's useful. We're starting to develop a Scheme vocabulary to reason
with streams. There's still a whole lot more we can do with them,
though. Recall one definition of streams which involved a {\bf
  starting element} and a {\bf function} to generate each new element
from previous one. This is a bit of a generalization of our usual
pattern of making streams. Let's look once again at {\tt ints-from}
and {\tt evens-from}:
\begin{scheme}
(define (ints-from n)
  (s-cons n (ints-from (+ n 1)))
(define (evens-from n)
  (s-cons n (evens-from (+ n 2))))
\end{scheme}
Let's look again with some pieces changed:
\begin{scheme}
(define (\textit{stream} n)
  (s-cons n (\textit{stream} (\textit{function} n))))
\end{scheme}
See the basic structure we've been using to create a stream? We take a
number $n$ as our starting line, and apply {\it function} to any
element to get the next element. In the first case, {\it function}
have been {\tt add-one} and in the second it might have been {\tt
  add-two}. Let's write a function {\tt s-make} that will {\bf create}
a stream for us given these two parts of our definition. This is a bit
complicated, so we'll go back to using \underline{ } temporarily.
\begin{scheme}
(define (s-make f n)
  (define (stream n)
    (\underline{        } \underline{  } (stream \underline{     })))
  (stream \underline{  }))
\end{scheme}
The blanks should be straightforward to fill in, but more importantly,
let's use this function to define some new streams, including the ones
we just discussed. Feel free to use {\tt add-one} and {\tt
  add-two}. You might find some of the simple functions you defined
earlier to be useful.
\begin{scheme}
(define ones          (s-make \underline{          } \underline{  }))
(define twos          (s-make \underline{          } \underline{  }))
(define ints          (s-make \underline{          } \underline{  }))
(define evens         (s-make \underline{          } \underline{  }))
(define powers-of-two (s-make \underline{          } \underline{  }))
\end{scheme}
An important thing to notice is that you just defined {\tt ints} and
{\tt evens} {\bf without} needing to define {\tt ints-from} and {\tt
  evens-from}. This function lets us make some streams very
conveniently! (Can we make {\tt fibs} with {\tt s-make}? What is the
general pattern missing that our earlier definition of {\tt fibs-from}
had.)

\subsection{Final Round: Complex Stream Operations}

\item There's yet more we can do to operate on streams. Can you think
  of anything? Can we do something (apply a function) to every element
  in a stream? Well, we don't know how long the stream goes on for,
  and everything we've worked with so far is an infinite stream. How
  in the world can we then do something to every element when there
  are an infinite number of elements? Hmm, what about making a new
  stream? What goes in the new stream? How about this: every element
  in the new stream is some function of an element in the old
  stream. That sounds like something we could possibly work with. So
  basically, the new stream is simply ``a function away'' from the old
  stream. Can we get an example? Ok:
\begin{interaction}
ones = 1,1,1,1,1,...
twos = ones * 2 = 1*2,1*2,1*2,1*2,... = 2,2,2,2,2,...
\end{interaction}
That makes sense. When we say {\tt ones * 2}, we're really saying take
every element of {\tt ones} and double it. The result is the stream of
twos. Let's assume we have a function \smap that takes a function $f$
and a stream of elements $(e_1,e_2,e_3,...)$ and returns a stream of
elements $(f(e_1),f(e_2),f(e_3),...)$. That's a mouthful, but here's
how we'd use it to do the above in Scheme:
\begin{scheme}
(define twos
  (s-map double ones))
\end{scheme}
That's all! Looks almost too simple, so let's actually {\it write}
\smap to see why.
\begin{scheme}
(define (s-map f s)
  (s-cons \underline{              }            \textit{;; apply function to first element}
          (s-map \underline{  } \underline{           })))  \textit{;; recurse on rest of stream}
\end{scheme}
Nice, we just defined another extremely useful function. You know what
comes next.
\item Feel free to use any of the simple functions you've already
  defined in combination with \smap to define the following familiar
  streams:
\begin{scheme}
(define twos-new
  (s-map \underline{          } twos))
(define threes
  (s-map add-one \underline{          }))
(define evens
  (s-map \underline{          } ints))
(define powers-of-two
  (s-map \underline{          } \underline{          }))
(define true-falses
  (s-map even? \underline{          }))
\end{scheme}
\item We just increased our repertoire yet again with \smap, which
  creates a new stream that's a function of another stream. What about
  making a new stream that's a function of {\bf two} other
  streams. Let's call it \smerge; it'll take a function $f$ and two
  streams, $(n_1,n_2,n_3,...)$ and $(m_1,m_2,m_3,...)$, and create a
  new stream $(f(n_1,m_1), f(n_2,m_2), f(n_3,m_3), ...)$. Once again,
  let's see an example before actually writing the function.
\begin{scheme}
(define twos
  (s-merge + ones ones))
\end{scheme}
\begin{interaction}
(s-generate twos 10)
=> (2 2 2 2 2 2 2 2 2 2)
\end{interaction}
Looks good, on to the definition:
\begin{scheme}
(define (s-merge f s1 s2)
  (s-cons \underline{                         }     \textit{;; apply f to first element of both streams}
    (s-merge \underline{  } \underline{         } \underline{         })))) \textit{;; recurse on rest of both streams}
\end{scheme}
\item Ok, you're probably bored of looking at our old friends again,
  so let's look at them in a new light. Define {\tt ints}, {\tt odds},
  and {\tt fibs} as merges of other streams. Feel free to use any
  stream or function we've defined so far. {\tt ints} and {\tt fibs}
  are slightly tricky, so don't feel bad it they take some time to
  work out. Hint: they're both recursively defined.
\begin{scheme}
\textit{;; use{\tt ones} and{\tt evens}}
(define odds
  (s-merge

                       ))
(define ints
  (s-cons 1
          (s-merge

                               )))
(define fibs
  (s-cons 0
          (s-cons 1
                  (s-merge

                                         ))))
\end{scheme}
\item Let's define our last two stream operators. \sfilt takes two
  arguments, a predicate function of one argument (like {\tt even?})
  and a stream and returns the stream of elements that ``pass'' the
  predicate function. That is, if the predicate function returns {\tt
    \#t} for an element in the input stream, that element should be
  part of the output stream.
\begin{scheme}
(define (s-filter f s)
  (if
      (s-cons
                                                        )
      (s-filter                                         )))
(define evens
  (s-filter                                             ))
\end{scheme}
\item This is it, our last stream operation. \sinter
\end{enumerate}

% \item Define a stream of 5 numbers (your favourite ones). Then, use
%   \sgen to to list the elements out.  This will be our first {\it
%     finite} stream! It will also be the first one that uses anything
%   other than a stream as the second argument to \scons.
% \begin{scheme}
% (define five-nums
%   (s-cons \underline{  }
%     (s-cons \underline{  }
%       (s-cons \underline{  }
%         (s-cons \underline{  }
%           (s-cons \underline{  } '())))))) ;; hey, it's an empty list!
% \end{scheme}
% \begin{interaction}
% (s-generate \underline{         } \underline{  })
% => \underline{                     }
% \end{interaction}
% You might be wondering why it is we can just stick a list in where
% we've usually stuck a stream. The reason is that \scons isn't too
% particular about what its arguments are so long as \scar and \scdr
% know how to deal with them.

\end{document}
