\documentclass[pdftex,11pt]{article}
\usepackage{fancyvrb}
\usepackage{color}
\usepackage[T1]{fontenc}
\usepackage[light,math]{iwona}
\usepackage{multicol}
\usepackage{hyperref}

\addtolength{\oddsidemargin}{-.875in}
\addtolength{\evensidemargin}{-.875in}
\addtolength{\textwidth}{1.75in}
\addtolength{\topmargin}{-.75in}
\addtolength{\textheight}{1.0in}
\newcommand{\spacew}[1]{\underline{\hspace{#1}}}
\setlength{\parindent}{0pt}

\newcommand{\cons}{{\tt cons}\ }
\newcommand{\car}{{\tt car}\ }
\newcommand{\cdr}{{\tt cdr}\ }
\newcommand{\lref}{{\tt list-ref}\ }

\begin{document}

\begin{center}
  \Huge{CS3S Review 1}\\
  \large Covers material through Quiz 4: Predicates
\end{center}

\DefineVerbatimEnvironment%
  {scheme}{Verbatim}
  {numbers=left,numbersep=5pt,
   frame=lines,framerule=.2mm,framesep=8pt,commandchars=\\\{\}}

\DefineVerbatimEnvironment%
  {answer}{Verbatim}
  {frame=single,framesep=8pt,framerule=.1mm,,commandchars=\\\{\}}

\DefineVerbatimEnvironment%
  {interaction}{Verbatim}
  {framesep=8pt,framerule=.1mm,,commandchars=\\\{\}}

\begin{enumerate}
\item What will happen if the following is typed into Scheme?  Some of
  these will be errors, so state what causes the error.
\vspace{.02in}
\renewcommand{\theenumii}{\Alph{enumii}}
\renewcommand{\labelenumii}{\theenumii}

\begin{minipage}{0.32\textwidth}
\begin{answer}
> (define j 4)
> (+ j 4)

\underline{                    }
\end{answer}
\end{minipage}
\hspace{.02in}
\begin{minipage}{0.32\textwidth}
\begin{answer}
> (define ag 100)
> (* ag ag)

\underline{                    }
\end{answer}
\end{minipage}
\hspace{.02in}
\begin{minipage}{0.32\textwidth}
\begin{answer}
> (define ag 100)
> (* ag 100)

\underline{                    }
\end{answer}
\end{minipage}\\

\begin{minipage}{0.32\textwidth}
\begin{answer}
> (define (f 4)
    (+ 4 f))

\underline{                    }
\end{answer}
\end{minipage}
\hspace{.02in}
\begin{minipage}{0.32\textwidth}
\begin{answer}
> (define f 4 (+ f 4))
> (f 4)

\underline{                    }
\end{answer}
\end{minipage}
\hspace{.02in}
\begin{minipage}{0.32\textwidth}
\begin{answer}
> (define (f g)
    (+ g 4))

\underline{                    }
\end{answer}
\end{minipage}

\begin{minipage}{0.32\textwidth}
\begin{answer}
> (define (g a b)
    (* a b))
> (define (row a b)
    (+ (g a a) (g b a)))
> (row 10 20)

\underline{                    }
\end{answer}
\end{minipage}
\hspace{.02in}
\begin{minipage}{0.32\textwidth}
\begin{answer}
> (define (foo be bo)
    (list (list 'be be)
          (list bo 'bo)))
> (foo 'a 10)

\underline{                    }
\end{answer}
\end{minipage}
\hspace{.02in}
\begin{minipage}{0.32\textwidth}
\begin{answer}
> (define (f? n)
    (if (= n 0) #t #f))
> (f? 0)

\underline{                    }
\end{answer}
\end{minipage}

\item Write the necessary code to get {\tt '(2 3)} from the following
  lists using only combinations of \car and \cdr. If it's not
  possible, just say so.

  \begin{itemize}
  \item {\tt '(1 2 3)} \underline{\hspace{3in}}
  \item {\tt '((2 3))} \underline{\hspace{3in}}
  \item {\tt '((2 (3 4)))} \underline{\hspace{3in}}
  \item {\tt '(1 2 3 4)} \underline{\hspace{3in}}
  \item {\tt '((2) (3))} \underline{\hspace{3in}}
  \item {\tt '((1) (2 3))} \underline{\hspace{3in}}
  \end{itemize}

\item Fill in the blanks with the necessary parentheses or list
  operations to get the result {\tt '(Jim is an idiot)}.
  \begin{itemize}
  \item {\tt (\spacew{1in} 'Jim \spacew{.7in} is \spacew{.7in} '(an
      idiot) \spacew{.7in})}
  \item {\tt (\spacew{1in} '(Jim) \spacew{.7in} '(is an) \spacew{.7in}
      'idiot \spacew{.7in})}
  \item {\tt (\spacew{1in} '(Jim is) \spacew{1in} '(not an) \spacew{.7in}
      '(idiot) \spacew{.7in})}
  \end{itemize}

\newpage

\item Write out all the steps in the evaluation of the following
  function call:
\begin{answer}
> (define (funcofmystery a b c)
    (+ a (* b (+ a c))))
> (funcofmystery (funcofmystery 10 20 30) 20 30)







\end{answer}

\item Show the output of the following snippets of code:

\begin{minipage}{0.4\textwidth}
\begin{answer}
> 'hello

\underline{                          }
\end{answer}
\end{minipage}
\hspace{.02in}
\begin{minipage}{0.5\textwidth}
\begin{answer}
> (define greet '(hello there))
> (append greet '(jim))

\underline{                          }
\end{answer}
\end{minipage}

\begin{minipage}{0.4\textwidth}
\begin{answer}
> '('yellow 'jello)

\underline{                         }
\end{answer}
\end{minipage}
\hspace{.02in}
\begin{minipage}{0.5\textwidth}
\begin{answer}
> (define jam (cons 'jim '(likes)))
> (append jam '(peanut butter))

\underline{                         }
\end{answer}
\end{minipage}

\begin{minipage}{0.4\textwidth}
\begin{answer}
> (define (foo b)
    (list b '(append b (food))))
> (foo 'no)

\underline{                         }
\end{answer}
\end{minipage}
\hspace{.02in}
\begin{minipage}{0.5\textwidth}
\begin{answer}
> (define greeting '(hello there))
> (define greeting '(jim))
> (append greeting greeting)

\underline{                         }
\end{answer}
\end{minipage}

\item Write a function that takes 2 numeric arguments and returns {\tt
    \#t} when they are both greater than 0. Write this once using {\tt
    if}, and once using {\tt cond}.
\vspace{2in}

\item Write a function that takes 3 arguments and returns the second
  argument if the first argument is 0 and the third argument if the
  first argument is 1. eg: {\tt (f 0 10 20) $\rightarrow$ 10, (f 1 10
    20) $\rightarrow$ 20}
\vspace{1in}

\item Write a function that takes 4 arguments and returns {\tt \#t} if
  each argument is the sum of the previous arguments: {\tt (f 1 1 2 3)
    $\rightarrow$ \#t}, {\tt (f 1 1 2 99) $\rightarrow$ \#f} (because
  99 is not the sum of 1 and 2) \vspace{1.25in}

\item \label{q:guessme} Explain what the following function does:
\begin{answer}
(define (guessme alpha beta gamma)
  (if (>= alpha 0)
      (if (>= beta 0) (>= gamma 0) #f)
      #f))
\end{answer}
\vspace{.5in}

\item For what inputs does this function return {\tt \#t}?
\begin{answer}
(define (guessme2 alpha beta gamma)
  (if (odd? alpha)
      (if (odd? beta) (not (odd? gamma)) (odd? gamma))
      (if (odd? beta) (odd? gamma) (not (odd? gamma)))))
\end{answer}
\vspace{.5in}

\item Rewrite \#\ref{q:guessme} using only {\tt and}.
\end{enumerate}
\end{document}
