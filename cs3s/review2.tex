\documentclass[pdftex,11pt]{article}
\usepackage{fancyvrb}
\usepackage{color}
\usepackage[T1]{fontenc}
\usepackage[light,math]{iwona}
\usepackage{multicol}
\usepackage{hyperref}

\addtolength{\oddsidemargin}{-.875in}
\addtolength{\evensidemargin}{-.875in}
\addtolength{\textwidth}{1.75in}
\addtolength{\topmargin}{-.75in}
\addtolength{\textheight}{1.0in}
\newcommand{\spacew}[1]{\underline{\hspace{#1}}}
\setlength{\parindent}{0pt}

\newcommand{\cons}{{\tt cons}\ }
\newcommand{\cond}{{\tt cond}\ }
\newcommand{\true}{{\tt \#t}\ }
\newcommand{\false}{{\tt \#f}\ }
\newcommand{\car}{{\tt car}\ }
\newcommand{\cdr}{{\tt cdr}\ }
\newcommand{\lref}{{\tt list-ref}\ }
\newcommand{\subseq}{{\tt subseq}\ }
\newcommand{\member}{{\tt member}\ }
\newcommand{\pos}{{\tt position}\ }

\begin{document}

\begin{center}
  \Huge{CS3S Review 2}\\
  \large Covers material through Quiz 7: Simple recursion
\end{center}

\DefineVerbatimEnvironment%
  {scheme}{Verbatim}
  {numbers=left,numbersep=5pt,
   frame=lines,framerule=.2mm,framesep=8pt,commandchars=\\\{\}}

\DefineVerbatimEnvironment%
  {answer}{Verbatim}
  {frame=single,framesep=8pt,framerule=.1mm,,commandchars=\\\{\}}

\DefineVerbatimEnvironment%
  {interaction}{Verbatim}
  {framesep=8pt,framerule=.1mm,,commandchars=\\\{\}}

\begin{enumerate}
\item Rewrite the following function to use a single \cond. In the
  space below it, explain what the function does.

\begin{answer}
(define (mystery a b c)
  (if (= a 0)
      (if (= b 0)
          (if (=c 0) #f #t)
          (if (=c 0) #t #f))
      (if (= b 0)
          (if (= c 0) #t #f)
          #f)))
\end{answer}

\vspace{1in}

\item Imagine a world in which \car and \cdr do not exist. Define them
  yourself using the functions \subseq and \lref as necessary.

\begin{minipage}{.47\textwidth}
\begin{answer}
(define (car l)

                                    )
\end{answer}
\end{minipage}
\begin{minipage}{.47\textwidth}
\begin{answer}
(define (cdr l)

                                    )
\end{answer}
\end{minipage}

\item Imagine you have been transported to a bizarro world where
  computers can only perform the most basic kind of addition: adding 1
  to a number. They do this with a Scheme function {\tt succ}, which
  takes a number $n$ and returns $n+1$. This enrages the computer
  scientist in you, and you realize it's up to you to save this world
  from total annihilation by inventing a better addition. Your
  function, {\tt bizarro-add}, takes two nonnegative integers and
  computes their sum using recursion and {\tt succ}.

\begin{answer}
(define (bizarro-add m n)




\hspace{6in})
\end{answer}

\item As soon as you finish writing that last function, you are
  violently transported to a new bizarro world. In this bizarro world,
  computers cannot perform multiplication. Unlike the last bizarro
  world, they {\bf can} do addition, which makes saving this world
  from destruction just a little bit easier for you. Your function,
  {\tt bizarro-mult}, takes two nonnegative integers and computes
  their product using recursion and {\tt +}.

\begin{answer}
(define (bizarro-mult m n)




\hspace{6in})
\end{answer}

\item You are getting disoriented from your bizarro jumps so you
  decide to take a vacation. Your travel agent tells you she'll give
  you a discount if you write her a Scheme function that determines
  the length of a list. Her version of Bizarro Scheme lacks a {\tt
    length} function, so it's up to you to do it recursively. {\tt
    length} takes a single argument, a list, and returns its length.

\begin{answer}
(define (bizarro-length l)




\hspace{6in})
\end{answer}

\item You are on vacation in Elbonia when you are captured by its
  secret service. They hold you at carrotpoint in front of a computer
  and tell you to write a Scheme function {\tt powers}, which takes a
  base $b$ and an exponent $e$ and returns all powers of $b$ from
  $b^1$ to $b^e$. eg: {\tt (powers 2 4) $\rightarrow$ (2 4 8
    16)}. They could not afford a version of Scheme with the {\tt
    expt} function, so you know your only way out of this is a {\bf
    recursive helper function}.

\begin{answer}
(define (powers base exp)
\hspace{6in})









\end{answer}

\newpage

\item The Elbonian nation is in the middle of a food shortage, so they
  want you to write a function to aid them called {\tt
    find-the-bacon} \footnote{Example courtesy of \underline{{\tt The
      Little MLer}}} . The function accepts a list and returns the
  position of {\tt 'bacon} in the list. Unfortunately, their
  impoverished version of Scheme lacks \member, so you'll have to write
  the function without it.

  eg. {\tt (find-the-bacon '(bacon 2 3 4)) $\rightarrow$ 1}, {\tt
    (find-the-bacon '(1 2 3 bacon)) $\rightarrow$ 4},\\
  {\tt (find-the-bacon '(1 2 3 4)) $\rightarrow$ 0}

  The last programmer (who, incidentally, died of starvation on the
  job) left a skeleton (no pun intended) for you to work with.

\begin{answer}
(define (find-the-bacon l)
  (cond ((null? l) \underline{   })
        ((equal? (car l) 'bacon) (+ 1 \underline{                   }))
        (else \underline{                    }))
\end{answer}

The function you just wrote doesn't really work. In what cases does it
fail to return the right answer? In these cases, what {\it does} it
return? Rewrite it to work for these cases:
\vspace{2in}

\item Write a (recursive) function that takes a list as an argument
  and returns {\tt \#t} if each element of the list is the sum of all
  of the previous elements: {\tt (f '(1 1 2 3)) $\rightarrow$ \#t},
  {\tt (f '(1 1 2 99)) $\rightarrow$ \#f} (because 99 is not the sum
  of 1, 1, and 2) \vspace{1.25in}


\end{enumerate}
\end{document}
